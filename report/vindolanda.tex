%% Commands for TeXCount
%TC:macro \cite [option:text,text]
%TC:macro \citep [option:text,text]
%TC:macro \citet [option:text,text]
%TC:envir table 0 1
%TC:envir table* 0 1
%TC:envir tabular [ignore] word
%TC:envir displaymath 0 word
%TC:envir math 0 word
%TC:envir comment 0 0
\documentclass[sigconf,authordraft]{acmart}

\author{Kieran John Knowles}
\affiliation{%
  \institution{Newcastle University}
  \city{Newcastle Upon Tyne}
  \country{United Kingdom}
}

%% Rights management information.  This information is sent to you
%% when you complete the rights form.  These commands have SAMPLE
%% values in them; it is your responsibility as an author to replace
%% the commands and values with those provided to you when you
%% complete the rights form.
% TODO: Copyright
\setcopyright{none}
\copyrightyear{2025}
\begin{document}

\title{Vindolanda VR}
% TODO: Video link

% TODO: Abstract
\begin{abstract}
\end{abstract}

%% The code below is generated by the tool at http://dl.acm.org/ccs.cfm.
%% Please copy and paste the code instead of the example below.
% TODO: CCS XML
\begin{CCSXML}
\end{CCSXML}

% \ccsdesc[500]{Do Not Use This Code~Generate the Correct Terms for Your Paper}
% \ccsdesc[300]{Do Not Use This Code~Generate the Correct Terms for Your Paper}
% \ccsdesc{Do Not Use This Code~Generate the Correct Terms for Your Paper}
% \ccsdesc[100]{Do Not Use This Code~Generate the Correct Terms for Your Paper}

% TODO: Keywords
% \keywords{Do, Not, Us, This, Code, Put, the, Correct, Terms, for,
%   Your, Paper}

% TODO: Teaser
% \begin{teaserfigure}
%   \includegraphics[width=\textwidth]{sampleteaser}
%   \caption{Seattle Mariners at Spring Training, 2010.}
%   \Description{Enjoying the baseball game from the third-base
%   seats. Ichiro Suzuki preparing to bat.}
%   \label{fig:teaser}
% \end{teaserfigure}

\maketitle

\section{Introduction}
% TODO: Introduction

\section{Background \& Related Work}
% TODO: Background Research}

\subsection{Historical Research}
% TODO: Historical Research}

\subsubsection{Site Visits}
% TODO: Site Visits}

\paragraph{Vindolanda}
% TODO: Vindolanda visit}

\paragraph{Hancock Museum}
% TODO: Hancock Museum visit}

\subsubsection{Literature}
% TODO: Literature}

\subsubsection{A Day in The Life}
% TODO: A Day in The Life at Vindolanda}

\subsection{Technical Research}
% TODO: Technical Research}

\subsubsection{Which Engine?}
% TODO: Which Engine}

Multiple engines were considered, each with their own benefits
and drawbacks as summarised in Table \ref{table:engine_compare}

Virtual reality is a significantly more demanding environment than other games.
While 1080p at 60Hz is a common target for flat screens, VR games must render
two screens at 90-120Hz each, with users often upscaling for enhanced clarity.
Consequently, performance is a crucial requirement for any VR game.

Given the short deadline of the project, industry adoption was also a major
consideration based on the assumption that a more widely adopted engine would
have more resources available to reduce development time. The total number of
games tagged as VR, detected as using a specific engine by
SteamDB (\url{https://steamdb.info/tech/}), and released
after May 2023 (to include current trends) were used to determine adoption.
This methodology has its limitations: by only using SteamDB as a source,
selection bias is introduced as non-steam releases such as itch.io and the Meta
Store are excluded. Additionally, SteamDB's engine detection can produce false
negatives, for example, Godot games can be packaged as a single executable which
will not be detected. Despite these limitations, this methodology was considered
good enough to get a high-level overview.
% TODO: Figure for false negative of Godot. Doesn't need to be a VR game

A smaller, but important, consideration was the engine's support for Git version
control. While it would have been possible to install and learn Perforce, doing
so would detract from development time, this made Unreal Engine a less practical
choice as it uses binary files for assets, which Git is poorly suited at
versioning.

\begin{table*}
  \caption{The advantages and disadvantages of the considered engines}
  \label{table:engine_compare}
  \begin{tabular}{llll}\toprule
    Feature & Unity & Unreal & Godot \\\midrule

    Language & C\# & Blueprints, C++ & GDScript, C++, C\#, etc. \\
    VR Support & De facto standard & Performance concerns & Yes \\
    Recent Games & $745 (70\%)$ & $351 (30\%)$ & $8 (<1\%)$ \\
    Version Control & Git, text & Perforce, binary & Git, text \\
    Linux Support & Yes & Editor unreliable & Yes \\

    \bottomrule
  \end{tabular}
\end{table*}

\subsubsection{Reference Material}
% TODO: Which other games were referenced as inspiration?}

\paragraph{Virtual Reality}
% TODO: VR inspiration. Mostly in how certain mechanics are implemented}

% TODO: Handling VR: Locomotion}
% TODO: Handling VR: Text}

\paragraph{Other Games}
% TODO: Other games, focus on visual inspiration}

\section{Design \& Implementation}
% TODO: Implementation}

\subsection{The Setting}

While Roman occupation of Britain spanned centuries, it was decided to set this
game during Emperor Hadrian's visit in 122 AD.
\cite[p.176]{danziger_hadrians_2006}, \cite[p.157]{moffat_wall_2009}
This setting gives the opportunity to teach about the rule of Hadrian himself
and the history leading to the construction of Hadrian's Wall, as well as other
events such as the disappearance of the ninth legion circa 120 AD.
% TODO: Source for 9th legion

% TODO: The setting}

\subsection{Educational Value}
% TODO: Educational Value. What makes a good educational game?


\section{Requirement Analysis}
% TODO: Requirement analysis

A requirement analysis using the MoSCoW method (Must-have, Should-have,
Could-have, and Won't-have) was performed to determine functional and
non-functional requirements for the game.

% TODO: Functional requirements}
\subsection{Functional requirements}

\paragraph{FR01 Recreate Vindolanda}
The game \textbf{must} include a recreation of Vindolanda during Hadrian's
rule.

\paragraph{FR02 Recreate Hadrian's Wall}
The game \textbf{should} include a recreation of Hadrian's wall during Hadrian's
rule.

\paragraph{FR03 Before and After}
The game \textbf{should} provide past and present representation of locations.

\subsection{Non-functional requirements}

\paragraph{NFR01 Performance}
The game \textbf{must} maintain 90FPS during typical gameplay.

\paragraph{NFR02 Linux Support}
The game \textbf{could} include native Linux support.

\section{Results \& Evaluation}
% TODO: Results and eval

\section{Conclusion \& Future Work}
% TODO: Conclusion and future work

% TODO: Acknowledgments
\begin{acks}

\end{acks}

\bibliographystyle{ACM-Reference-Format}
\bibliography{vindolanda}

\appendix

\end{document}
