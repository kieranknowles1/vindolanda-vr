\documentclass[12pt, a4paper]{report}

\usepackage{xltabular}
\usepackage{booktabs}

% TODO: Remove once no longer needed
\usepackage{todo}

\usepackage{hyperref}

% Multiline left-aligned column
\newcolumntype{L}{>{\raggedright\arraybackslash}X}

\title{Vindolanda VR}
\author{Kieran John Knowles}
\date{2025}

\begin{document}
\maketitle

\chapter*{Abstract}
\todo{Abstract}

\chapter*{Acknowledgments}
\todo{Acknowledgments}

\tableofcontents
\listoftables
\listoffigures

\todo{Use roman numeral page numbers for preliminary pages}

\chapter{Introduction}
\todo{Introduction}

\chapter{Background Research}
\todo{Background Research}

\section{Historical Research}
\todo{Historical Research}

\section{Technical Research}
\todo{Technical Research}

\subsection{Which Engine?}
\todo{Which Engine}

Multiple engines were considered, each with their own benefits
and drawbacks as summarised in Table \ref{table:engine_compare}

Virtual reality is a significantly more demanding environment than other games.
While 1080p at 60Hz is a common target for flat screens, VR games must render
two screens at 90-120Hz each, with users often upscaling for enhanced clarity.
Consequently, performance is a crucial requirement for any VR game.

Given the short deadline of the project, industry adoption was also a major
consideration based on the assumption that a more widely adopted engine would
have more resources available to reduce development time.

\begin{table}
\caption{The advantages and disadvantages of the considered engines}
\label{table:engine_compare}\begin{tabularx}{\textwidth}{cLLL}\toprule
Feature & Unreal & Unity & Godot \\\midrule

Language & C\# & Blueprints, C++ & GDScript, C++, C\#, etc. \\
VR Support & De facto standard & Yes, performance could be a concern & Yes \\
Version Control & Git, text & Perforce, binary & Git, text \\
Linux Support & Yes & Editor unreliable & Yes \\

\bottomrule\end{tabularx}
\end{table}

\chapter{Implementation}
\todo{Implementation}

\section{Requirement Analysis}
\todo{Requirement Analysis}

\newcommand{\header}{\toprule{}ID & Name & Priority & Description \\\midrule}
\newcounter{reqindex}
\newcommand{\reqtype}{FR}
\newcommand{\requirement}[3]{\reqtype\stepcounter{reqindex}\arabic{reqindex} & #1 & #2 & #3 \\}

\begin{xltabular}{\textwidth}{cccL}
\caption{Requirements using the MoSCoW method}\\\header\endfirsthead
\header\endhead

\renewcommand{\reqtype}{NFR}\setcounter{reqindex}{0}
\requirement{Performance}{Must have}
{The game \textbf{must} maintain 90FPS during typical gameplay}

\bottomrule

\end{xltabular}

\todo{Bibliography}

\todos
\end{document}
